\bluepage{Video player in OpenGL}

\begin{frame}[fragile]
  \frametitle{Playing video in OpenGL}
  \begin{itemize}
    \item A window has to be created using 3th party libraries (SDL2, GLUT, Qt).
    \item A video has to be decoded outside of OpenGL using other libraries (OpenCV, ffmpeg, ...).
    \item Current frame has to be stored in OpenGL texture or buffer in GPU memory.
    \item A precise time has to be used for correct synchronisation of video frames and rendering frames (std::chrono).
    \item A shader program has to contain some king of fullscreen quad program.
\end{itemize}
\end{frame}

\begin{frame}[fragile]
\frametitle{SDL2 Window}
  OpenGL requires a rendering context.
  There are 3 main context parameters (version, profile and flags) that can be set using SDL2.
  {\scriptsize
  \begin{minted}[frame=lines]{c++}
SDL_Init(SDL_INIT_VIDEO);//init. video
this->_window = SDL_CreateWindow("sdl2",0,0,width,heigth,
    SDL_WINDOW_OPENGL|SDL_WINDOW_SHOWN);

unsigned profile = SDL_GL_CONTEXT_PROFILE_CORE;//context profile
unsigned flags   = SDL_GL_CONTEXT_DEBUG_FLAG;//context flags
SDL_GL_SetAttribute(SDL_GL_CONTEXT_PROFILE_MASK ,profile         );
SDL_GL_SetAttribute(SDL_GL_CONTEXT_FLAGS        ,flags           );


std::vector<uint32_t>versions={450,440,430,420,410,400,330};
for(auto x:versions){
  SDL_GL_SetAttribute(SDL_GL_CONTEXT_MAJOR_VERSION, x/100    );
  SDL_GL_SetAttribute(SDL_GL_CONTEXT_MINOR_VERSION,(x%100)/10);
  this->_context = SDL_GL_CreateContext(this->_window);//create context
  if(this->_context!=nullptr)break;
}

glewInit();//initialisation of gl* functions
  \end{minted}
  }
\end{frame}

\begin{frame}[fragile]
\frametitle{OpenCV video decoding}
  OpenCV library has a lot of functions that deals with images and videos.
  cv::VideoCapture object is able to open video and load frames.
  {\scriptsize
  \begin{minted}[frame=lines]{c++}
#include<opencv2/core/core.hpp>
#include<opencv2/highgui/highgui.hpp>

class Video{
  protected:
    cv::Mat bgr_frame;
    cv::VideoCapture _video;
    ...
};

Video::Video(std::string fileName){
  this->_opened=this->_video.open(fileName);
  if(!this->_opened){...}
  this->_frameCount = this->_video.get(CV_CAP_PROP_FRAME_COUNT);
  this->_fps        = this->_video.get(CV_CAP_PROP_FPS);
  this->_width      = this->_video.get(CV_CAP_PROP_FRAME_WIDTH);
  this->_height     = this->_video.get(CV_CAP_PROP_FRAME_HEIGHT);
}
//read current frame and move to next
void* Video::getData(){
  ...
  if(this->_video.read(bgr_frame))return nullptr;
  return this->bgr_frame.data;
  ...
}
  \end{minted}
  }
\end{frame}

\begin{frame}[fragile]
\frametitle{Precise timers in std::chrono}
  std::chrono - C++ std library that provides high precise timers.
  {\scriptsize
  \begin{minted}[frame=lines]{c++}
#include<chrono>
#include<ctime>

template<typename TYPE>
class Timer{
  protected:
    std::chrono::time_point<std::chrono::high_resolution_clock>_start;
    std::chrono::time_point<std::chrono::high_resolution_clock>_last ;
  public:
    Timer(){
      this->reset();
    }
    void reset(){
      this->_start = std::chrono::high_resolution_clock::now();
      this->_last  = this->_start;
    }
    TYPE elapsedFromStart(){
      auto newTime = std::chrono::high_resolution_clock::now();
      std::chrono::duration<TYPE> elapsed = newTime - this->_start;
      this->_last = newTime;
      return elapsed.count();
    }
    ...
};
  \end{minted}
  }
\end{frame}

\begin{frame}[fragile]
\frametitle{Video frame loading into texture}
  Video frames and window frames has to be correctly synchronized.
  {\scriptsize
  \begin{minted}[frame=lines]{c++}
void draw(Data*data){
  glClear(GL_COLOR_BUFFER_BIT|GL_DEPTH_BUFFER_BIT);

  ...
  float  newTime = data->timer.elapsedFromStart();
  if((newTime-data->lastFrameTime)>1./data->video->getFps()){
    if(data->running){
      glBindTexture(GL_TEXTURE_2D,data->frame->getId());
      glTexImage2D(GL_TEXTURE_2D,0,GL_RGB,
        data->video->getWidth(),
        data->video->getHeight(),0,
        GL_BGR,GL_UNSIGNED_BYTE,data->video->getData());
      glTexParameteri(GL_TEXTURE_2D,GL_TEXTURE_MIN_FILTER,GL_LINEAR);
      glTexParameteri(GL_TEXTURE_2D,GL_TEXTURE_MAG_FILTER,GL_LINEAR);
      data->lastFrameTime = newTime;
    }
  }

  ...
}
  \end{minted}
  }
\end{frame}



\begin{frame}[fragile]
\frametitle{Fullscreen quad}
  In order to render a frame, there has to be textured rectangle.
  There are many ways of creating of rectangle (buffers, geometry shaders, vertex shaders,...).
  The simplest way employ vertex shaders.
  {\scriptsize
  \begin{minted}[frame=lines]{glsl}
#version 450

//texture coordinates
out vec2 vCoord;

void main(){
  //compute texture coordinates using gl_VertexID
  vec2 coord = vec2(gl_VertexID%2,gl_VertexID/2);

  //compute clips space vertex positions
  gl_Position = vec4(-1+2*coord,1,1.);

  //send texture coordinates to fragment shader
  vCoord = vec2(coord.x,1-coord.y);
}
  \end{minted}
  }
  {\scriptsize
  \begin{minted}[frame=lines]{c++}
//invocate vertex shader 4 times and construct triangle strip
glDrawArrays(GL_TRIANGLE_STRIP,0,4);
  \end{minted}
  }
\end{frame}

\begin{frame}[fragile]
\frametitle{Fullscreen quad}
  {\scriptsize
  \begin{minted}[frame=lines]{glsl}
#version 450

//output for color
layout(location=0)out vec4 fColor;

//texture
layout(binding=0)uniform sampler2D frame;

//texture coordinates
in vec2 vCoord;

void main(){
  //address texture and return color
  fColor = texture(frame,vCoord);
}
  \end{minted}
  }
  {\scriptsize
  \begin{minted}[frame=lines]{c++}
//active texture unit 0
glActiveTexture(GL_TEXTURE0);

//bind texture frame to texture unit 0
glBindTexture(GL_TEXTURE_2D,frame);
  \end{minted}
  }
\end{frame}


