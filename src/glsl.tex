\begin{frame}
\frametitle{Jazyk GLSL}
  \begin{itemize}
  \item GLSL - OpenGL Shading Language
  \item Slouží pro popis programů, které běží na GPU
  \item Je odvozený od C
  \item Neobsahuje rekurzi, třídy, výjimky, std knihovny
  \item Obsahuje vektorové a maticové typy, vestavěné funkce, vestavěné proměnné, synchronizační funkce, typové kvalifikátory, rozšířené adresování vektorů
  \item Každý shader musí obsahovat main funkci
  \item Každá main funkce je vykonávána v mnoha instancích v dané části pipeline
  \item Některé části pipeline mají speciální nastavení
  \end{itemize}
\end{frame}

\begin{frame}[fragile]
\frametitle{Jazyk GLSL - příklad typy}
		{\scriptsize
		\begin{minted}[frame=lines]{glsl}
#version 450

void main(){
  //32 bit integer
  int a;
  //32 bit unsigned integer
  uint b;
  //32 bit float
  float c;
  //vector of 4 ints
  ivec4 d;
  //vector of 3 floats
  vec3 e = vec3(1,2,3);
  //matrix 3x3 of floats
  mat3 m;
  //zeroth element of e
  e[0] == e.x == e.r;
  //swizzling 
  vec2 f = e.xy; // (1,2)
  f = e.zz; // (3,3)
  //matrix vector multiplication
  e = m*e;
  //constructing ivec4 from vec3 and scalar
  d = ivec4(c,4);
  d = ivec4(c.xx,c.yy);
}
		\end{minted}
		}
\end{frame}

\begin{frame}[fragile]
\frametitle{Jazyk GLSL - vestavěné funkce}
{\tiny
\begin{minted}[frame=lines]{glsl}
abs acos acosh asin asinh atan atanh ceil cos cosh degrees exp exp2 floor fract inversesqrt
log log2 max min mod modf pow radians round roundEven sign sin sinh sqrt tan tanh trunc
clamp cross distance dot floatBitsToInt floatBitsToUint fma frexp intBitsToFloat isinf
isnan ldexp length mix normalize smoothstep step

packDouble2x32 packSnorm4x8 packUnorm2x16 packSnorm2x16
packUnorm4x8 uintBitsToFloat unpackDouble2x32 unpackSnorm4x8 unpackUnorm2x16 
unpackSnorm2x16 unpackUnorm4x8 packHalf2x16 unpackHalf2x16

all any bitCount bitfieldExtract bitfieldInsert bitfieldReverse determinant equal
faceforward findLSB findMSB greaterThan greaterThanEqual imulExtended inverse lessThan
lessThanEqual matrixCompMult not notEqual outerProduct reflect refract transpose uaddCarry
umulExtended usubBorrow 

textureSize textureQueryLod texture textureProj textureLod 
textureOffset texelFetch texelFetchOffset textureProjOffset textureLodOffset textureProjLod 
textureProjLodOffset textureGrad textureGradOffset textureProjGrad textureProjGradOffset 
textureGather textureGatherOffset textureGatherOffsets textureQueryLevels

dFdx dFdy fwidth
interpolateAtCentroid interpolateAtOffset interpolateAtSample

noise1 noise2 noise3 noise4

EmitStreamVertex EndStreamPrimitive EmitVertex EndPrimitive

barrier memoryBarrier 
memoryBarrierAtomicCounter memoryBarrierBuffer memoryBarrierImage memoryBarrierShared 
groupMemoryBarrier
imageSize

atomicAdd atomicMin atomicMax atomicAnd atomicOr atomicXor atomicExchange atomicCompSwap

imageSize imageLoad imageStore imageAtomicAdd imageAtomicMin imageAtomicMax
imageAtomicAnd imageAtomicOr imageAtomicXor imageAtomicExchange imageAtomicCompSwap
\end{minted}
}
\end{frame}

\begin{frame}[fragile]
\frametitle{Jazyk GLSL - kvalifikátory proměnných}
{\tiny
\begin{minted}[frame=lines]{glsl}
#version 450

//promenna a je plnena predchazejici shader stage nebo
//z bufferu GL_ARRAY_BUFFER
//read only
in vec4 a;

//hodnota b bude viditelna v dalsi shader stage nebo
//ve framebufferu
//write only
out vec4 b;

//hodnoty promennych a,b se meni s kazdou invokaci shaderu

//promenna m je ulozena v konstantni pameti, lze ji vycist 
//ve vsech shader stage
//read only
uniform mat4 m;

//hodnota m je nemenna pro vsechny invokace shaderu

//promenna d je typu textury (obrazek), je to opaque type, ktery
//lze cist jen pomoci specialnich funkci
//read only
uniform sampler2D d;

//promenna e je lokalni, je ulozena v registru, kazda 
//invokace shaderu ma svoji vlastni
vec4 e = vec4(0,1,2,3);

void main(){
  b = e + m * texture(d,a.xy);
}
\end{minted}
}
\end{frame}

\begin{frame}[fragile]
\frametitle{Jazyk GLSL - důležité vestavěné proměnné}
Vertex Shader obsahuje několik důležitých vestavěných proměnných
{\scriptsize
\begin{minted}[frame=lines]{glsl}
#version 450

void main(){
  //vystupni promenna, sem se zapisuje pozice vrcholu
  //po perspektivni projekci
  vec4 gl_Position;

  //cislo vrcholu
  in int gl_VertexID;
  //cislo instsance
  in int gl_InstanceID;
  //cislo draw callu
  in int gl_DrawID;
  //velikost primitiva typu bod
  out float gl_PointSize;
  out float gl_ClipDistance[];
  out float gl_CullDistance[];
}

\end{minted}
}
\end{frame}

\begin{frame}[fragile]
\frametitle{Jazyk GLSL - důležité vestavěné proměnné}
Fragment Shader obsahuje několik důležitých vestavěných proměnných.
Výstup fragment shaderu si specifikuje programátor pomocí vlastní výstupní proměnné.
{\scriptsize
\begin{minted}[frame=lines]{glsl}
#version 450

//vlastni vystup, namapuje se na 0. barevny framebuffer
layout(location=0)out vec4 fColor;

void main(){
  //coordinaty fragmentu ve viewportu
  in vec4 gl_FragCoord;

  //vznikl fragment z privracene strany primitiva
  in bool gl_FrontFacing;

  //pro modifikaci hloubky fragmentu
  //pri zapisu vypina brzky depth test
  out float gl_FragDepth;

  in float gl_ClipDistance[];
  in float gl_CullDistance[];
  in int gl_PrimitiveID;

}

\end{minted}
}
\end{frame}

