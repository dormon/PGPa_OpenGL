\begin{frame}[fragile]
\frametitle{Kompletní příklad tvorba hier. z-bufferu}
  \begin{itemize}
    GLSL - CreateHierarchy:
    {\tiny
    \begin{minted}[frame=lines]{glsl}
    //vertex shader////////////////////////////////////////////////////////////
    #version 430
    void main(){
      gl_Position=vec4(0);
    }
    //geometry shader//////////////////////////////////////////////////////////
    #version 430
    layout(points)in;
    layout(triangle_strip,max_vertices=4)out;
    void main(){
      gl_Position=vec4(-1,-1,0,1);EmitVertex();
      gl_Position=vec4(+1,-1,0,1);EmitVertex();
      gl_Position=vec4(-1,+1,0,1);EmitVertex();
      gl_Position=vec4(+1,+1,0,1);EmitVertex();
    }
    //fragment shader///////////////////////////////////////////////////////////
    #version 430
    layout(binding=0)uniform sampler2D Last;//last mipmap
    layout(location=0)out vec2 fDepth;//output depth
    ivec2 Coord=ivec2(gl_FragCoord.xy);//coordinates
    void main(void){
      vec2 A=texelFetch(Last,Coord*2+ivec2(0,0),0).xy;
      vec2 B=texelFetch(Last,Coord*2+ivec2(1,0),0).xy;
      vec2 C=texelFetch(Last,Coord*2+ivec2(0,1),0).xy;
      vec2 D=texelFetch(Last,Coord*2+ivec2(1,1),0).xy;
      fDepth=vec2(min(min(A.x,B.x),min(C.x,D.x)),max(max(A.y,B.y),max(C.y,D.y)));
    }
    \end{minted}
    }
  \end{itemize}
\end{frame}

\begin{frame}[fragile]
\frametitle{Kompletní příklad tvorba hier. z-bufferu}
  \begin{itemize}
    Aplikace - inicializace:
    {\scriptsize
    \begin{minted}[frame=lines]{c++}
    glGenTextures(1,&Depth);
    //textura obsahujici minimalni a maximalni hloubku
    glBindTexture(GL_TEXTURE_2D,Depth);
    glTexParameteri(GL_TEXTURE_2D,GL_TEXTURE_MAG_FILTER,GL_NEAREST);
    glTexParameteri(GL_TEXTURE_2D,GL_TEXTURE_MIN_FILTER,GL_NEAREST_MIPMAP_NEAREST);
    int Size=WSize;
    int Level=0;
    while(Size>0){//smycka pres levely
      glTexImage2D(GL_TEXTURE_2D,Level++,GL_RG32F,Size,Size,0,GL_RG,
        GL_FLOAT,NULL);//alokace vrstvy
      Size/=2;
    }
    //render buffer s hloubkou
    glGenRenderbuffers(1,&RBO_Depth);
    glBindRenderbuffer(GL_RENDERBUFFER,RBO_Depth);
    glRenderbufferStorage(GL_RENDERBUFFER,GL_DEPTH_COMPONENT,WSize,WSize);

    glGenFramebuffers(1,&FBO);//vygenerujeme nazev pro FBO
    \end{minted}
    }
  \end{itemize}
\end{frame}

\begin{frame}[fragile]
\frametitle{Kompletní příklad tvorba hier. z-bufferu}
  \begin{itemize}
    Aplikace - render2 - tvorba hierarchie:
    {\tiny
    \begin{minted}[frame=lines]{c++}
    glUseProgram(CreateHierarchy);
    int Level=1;
    int ActSize=WSize/2;
    glBindFramebuffer(GL_FRAMEBUFFER,HFBO);//bind framebuffer
    glActiveTexture(GL_TEXTURE0);//activate tex. unit 0
    glBindTexture(GL_TEXTURE_2D,Depth);//bind depth texture to tex. unit 0
    while(ActSize>0){//while there are
      glViewport(0,0,ActSize,ActSize);//set viewport
      glTexParameteri(GL_TEXTURE_2D,GL_TEXTURE_BASE_LEVEL,Level-1);//starting mipmap level
      glTexParameteri(GL_TEXTURE_2D,GL_TEXTURE_MAX_LEVEL,Level-1);//max mipmap level
      glFramebufferTexture2D(GL_FRAMEBUFFER,GL_COLOR_ATTACHMENT0,GL_TEXTURE_2D,Depth,Level);
      GLenum DrawBuffers[]={GL_COLOR_ATTACHMENT0};
      glDrawBuffers(1,DrawBuffers);
      glDrawArrays(GL_POINTS,0,1);
      Level++;//increment level of mipmap
      ActSize/=2;//actual size of mipmap
    }
    glBindFramebuffer(GL_FRAMEBUFFER,0);//unbind framebffer
    glTexParameteri(GL_TEXTURE_2D, GL_TEXTURE_BASE_LEVEL, 0);//starting mipmap level
    glTexParameteri(GL_TEXTURE_2D, GL_TEXTURE_MAX_LEVEL, Level-1);//max mipmap level
    glViewport(0,0,WSize,WSize);//reset viewport
    \end{minted}
    }
  \end{itemize}
\end{frame}


